\documentclass{article}
\usepackage{mathtools}
\usepackage{amssymb}
\usepackage{amsthm}
\usepackage{physics}
\usepackage{stmaryrd}
\usepackage{bbm}
\usepackage{hyperref}
\hypersetup{
    colorlinks=true,
    linkcolor=blue,
    }
\usepackage{biblatex}
\addbibresource{refs.bib}

\begin{document}

\tableofcontents

\section{Corpus to vectors}

\subsection{Preprocessing}

Stopwords are commonly used words in a language that are considered to have little or no significance in determining the meaning of a text.
`not' is not a stopword.

\subsection{One hot pot}

\subsection{Bag of Words}

\subsection{tf-idf}

The BoW (Bag of Words) model assumes that the importance of a term is directly proportional to
the number of its appearance in the document, this can easily be misleading
when the most common words are `stopwords'. (But it really depends on
the algorithm that we are going to use afterwards, if we do perceptron, it wouldn't matter.)

tf-idf (term frequency - inverse document frequency) first considers
the whole of a corpus, it assumes a term too frequent in the corpus has little
information. Then it considers the importance of a term in a document
as the frequency of the term in this document times a scalar
representing its information in the corpus.

$$
\begin{aligned}
\mathrm{tfidf}(\mathrm{term}) & = \mathrm{tf}(\mathrm{term}) \times \mathrm{idf}(\mathrm{term}) \\
\mathrm{tf}(\mathrm{term}) & = \frac{\# \text { of times term appears in document }}{\# \text { of terms in document }} \\
\mathrm{idf}(\mathrm{term}) & =\ln \left(\frac{\# \text { of documents }}{\# \text { of documents in corpus with term }}\right)
\end{aligned}
$$

Other possible choices:

\begin{itemize}
  \item A term can be many (succesive) words (n-gram) (e.g. to
  take into account of negations before a word)
  \item The exact formulae can be changed while the same idea remains
\end{itemize}

\paragraph{On the implementation} Il faut éviter utiliser \verb|for|, et
utiliser les fonctions numpy à la place; on utilise les matrices sparse pour
gagner de la mémoire et parfois l'éfficacité.

\paragraph{Observations}
\begin{itemize}
  \item Curse of dimensionality
  \item Preprocessing is important
\end{itemize}

\section{Treating vectors}

\subsection{Distances}

We can use euclidean distance, or we can use the distance concerning
only the angle between 2 vectors (so the `difference of lengths' of documents
is ignored).

$$
\begin{aligned}
&\text{(cosine similarity)}S(A, B) :=
\cos (\theta)=\frac{A \cdot B}{\|A\|\|B\|}\\
&\text{(cosine distance)}D(A, B) :=1-S(A, B)
\end{aligned}
$$

\subsection{Dimensionality reduction}

\begin{itemize}
  \item PCA (not useful for us, every new word (if not a stopword) can not be ignored and their linear combinations have no meaning)
  \item t-SNE (only useful for visulization?)
\end{itemize}

\subsection{Topological data analysis}

Try citation \cite{Zhu_2013}

\subsection{k nearest neighbour}

One predict the information associated with a vector by the majority
of information associated with its neighbours (within k nearest).

\paragraph{On implementation} \begin{itemize}
  \item We favor the nearer information when there is a tie.
  \item Python is not a well typed language, it is so easy to have
  ridiculous bugs.
  \item cosine distance works much better than euclidean distance.
\end{itemize}

\paragraph{Positive} \begin{itemize}
  \item Easy to implement
\end{itemize}

\paragraph{Negative} \begin{itemize}
  \item Supervised
\end{itemize}

\subsection{Perceptron}

\paragraph{On the implementation}
% \begin{itemize}
% \end{itemize}

\printbibliography

\end{document}